\section{Installation}
\writer{Cosmin}
\label{sec:installation}

This section discusses the installation of the product and presents the requirements and main steps
necessary to make the system work.

\subsection{Prerequisites}
Our product is built as a series of extensions for the Eclipse Development Platform and requires
Java. Furthermore, ePNK 0.9.4 has to be set up in Eclipse and the Java3D 1.5.1 library has to be
installed.

First of all, \textit{Eclipse 4.2.0} (Juno) has to be installed on the computer. The installation
can be done by following the instructions found in the Eclipse Install Guide
\footnote{http://wiki.eclipse.org/Eclipse/Installation} or by navigating to
\url{http://www.eclipse.org/downloads}.

Secondly, \textit{Java Runtime Environment 1.6} has to be installed. Setting up the Java
environment can be done by following the instructions found on the Java homepage,
\url{http://www.java.com} or by navigating directly to the Java Downloads page
\footnote{http://java.com/en/download/manual.jsp}.

The \textit{ePNK} can be installed, as an Eclipse extension, by following the details provided on
its homepage, \url{http://www2.imm.dtu.dk/~eki/projects/ePNK/install-details.html}. For more
details, ePNK's manual \footnote{http://orbit.dtu.dk/getResource?recordId=275136} can also be
consulted.

The last requirement, \textit{Java3D} library, can be set up by following the instructions on the
product's home page, \url{http://java3d.java.net/} or on the install guide page
\footnote{http://download.java.net/media/java3d/builds/release/1.5.1/README-download.html}.

\subsection{Product installation}
As mentioned before, the product has to be installed as an Eclipse plugin. The final version of
the product will be available as an Eclipse plugin that can be installed via an update 
site (or, for offline installation, as a site of files that need to be copied in the Eclipse plugins 
directory). The plugin will be available only after Eclipse has been restarted, if it was running.

During the development period, the product is available as a set of projects\footnote{The latest
version is available at https://svn.imm.dtu.dk/se2/svn/e12-groupE/project/} that have to be
downloaded locally and imported into the Eclipse Development Environment. After opening the projects, a
Runtime Eclipse Workbench has to be started by selecting one of the projects and then by clicking on
the \textit{Run} command.
