\section{Introduction}
\label{sec:introduction}
\writer{Anders}

This is a handbook for the \index{ePNS}``group \textbf{e} \textbf{P}etri \textbf{N}et \textbf{S}imulator'': \epns.
The purpose of \epns{} is to facilitate \index{track based simulation}track based simulations of Petri nets.
A track based simulation is understood as a simulation of a world, where objects move along tracks,
following a set of rules given by the Petri net. The objects can be simulated as trains on tracks, cars on roads,
or any other object following a predefined route.

\subsection{What can be simulated?}

Petri nets are, with the addition of \index{inhibitor arc}inhibitor 
arcs\footnote{Inhibitor arcs prevents transitions from fireing unless the associated place is empty.
Inhibitor arcs are not implemented by \epns{} yet.},
Turing complete,
meaning that they can implement any computer program.
We will not formally prove this property, as it is out of scope for this
assignment.
But it is not hard to understand why: the inhibitor arc makes it possible to
implement a NAND gate (see figure \ref{fig:petri-nand}).
All binary operations can be expressed using the NAND operation, some examples:
\begin{align*}
    \overline{A} &= \overline{A \cdot A}               
	                & NAND(A, A) \\
    A \cdot B    &= \overline{\overline{A \cdot B}}
    				& NAND(NAND(A, B), NAND(A, B))\\
    A + B        &= \overline{\overline{A + B}} = \overline{\overline{A} \cdot \overline{B}} 
    				& NAND(NAND(A, A), NAND(B, B)) \\
    A \otimes B  &= A \cdot \overline{B} + B \cdot \overline{A}
    			  = \overline{\overline{A} \cdot \overline{B}} \cdot \overline{A \cdot B}
    				& \text{use previous formulas \ldots}
\end{align*}

\begin{figure}[htp]
\begin{center}
\begin{petri}
    \node at (0, 0) [inputplace] (inputA) {$A$};
    \node at (0, 3) [inputplace] (inputB) {$B$};
	\node at (2, 1.5) [transition] (andtransition) {$T_1$};
	\node at (4, 1.5) [place] (andoutput) {$A B$};
	\node at (6.5, 1.5) [inhibitor] (inhibitor) {};
	\node at (5.6, 1.8) {inhibitor};
	\node at (7, 1.5) [transition] (nandtransition) {$T_2$};
	\node at (9, 1.5) [place] (nandoutput) {$\overline{A B}$};
	\draw [->,arc] (inputA) to [out=60,in=200] (andtransition);
	\draw [->,arc] (inputB) to [out=-60,in=160] (andtransition);
	\draw [->,arc] (andtransition) to [out=0,in=180] (andoutput);
	\draw [->,arc] (andoutput) to [out=0,in=180] (inhibitor);
	\draw [->,arc] (nandtransition) to [out=0,in=180] (nandoutput);
	%\draw [->,arc] (connection) to [out=210,in=330] (track);
	%\draw [->,arc] (semaphore) -- (connection);
\end{petri}
\caption{Petri net for a NAND gate}
\label{fig:petri-nand}
\end{center}
\end{figure}

Consequently not all Petri nets are suited for track based simulations.
The main concept of Petri nets are that transitions consume tokens from places connected with inbound arcs,
and produces tokens in places connected with outbound arcs.

Therefore don't expect any Petri net to produce a logical or useful 3D
simulation. Not even if it doesn't include inhibitor arcs.

In \epns{}, if a token is produced in a place associated with a track,
the token itself will be associated with a moving object.
Therefore, Petri nets that produce the same amount of tokens as it consume
will be most suited for \epns{} simulations.
This is not a mandatory property, but it ensures that the moving objects don't disappear or
appear out of the blue\footnote{In a production line simulation it might be on purpose.
For example, four wheels and a car body disappears and a car arises.}.

Other places can be associated with objects that show the state of the underlying logic.
In the train simulation world, they could be traffic lights or switches.

However, other places should be designated to receive input from the user.
Such places might not be part of the initial logic, but should be added where input is needed.

\subsection{How to use the handbook?}

In the \nameref{sec:installation} section, the necessary steps to have a running
installation are reviewed. An installation includes the Eclipse IDE and additional modules. 

For first time users, the section \ref{sec:tutorial} (\nameref{sec:tutorial}) will cover 
the steps needed to create a simple Petri net simulation. These steps include:
\begin{itemize}
  \item Creating the Petri net.
  \item Laying out the tracks and the positions of input and output objects in the associated geometry model.
  \item Selecting the appearance of the elements.
  \item Creating the configuration for the simulation.
  \item Running the simulator.
\end{itemize}

For more experienced users, the section \ref{sec:userguide} (\nameref{sec:userguide}) will provide
detailed information on the various steps in creating more complex simulations. 

Please consult the index and glossary for information on specific subjects.

\subsection{Basic functionality of ePNS}
\writer{Cosmin}
In order to use \epns, the application user starts by using the \textbf{Petri net Editor} to create
and configure the Petri net and the \textbf{Animations} that are executed during the simulation.
These will be later loaded by the \textbf{Graphical Simulator}, which, after being started, will
simulate the movement of tokens through the Petri net and will execute the configured animations,
resulting a visual representation of the Petri net's simulation on the screen.

Then, using the \textbf{GeometryEditor}, the \textbf{Geometry} would be created. It would be later
used by the \textbf{Graphical Simulator} in order to know how (on which paths) to move objects/token
representations and where to place different objects in the 3D space\footnote{Even though the
geometry is specified in a 2D space, during the simulation, all the object representations will be
drawn as 3D objects moving on a plane.}.

The \textbf{Graphical Simulator} also requires information about the \textbf{Appearance}, in order
to know how to represent tokens, tracks and other objects during the simulation. It is a simple
editor that connects labels (keys) with 3D Models(vrml, png, jpg \ldots), textures or just plain
data (Colors, Shapes).
  
The last step is to create a \textbf{Configurator} that connects the previously created
configurations and allows the user to start the graphical simulation. When started, the
\textbf{Graphical Simulator} reads the state of the simulation from the \textit{Petri net}. This read
state does not include exact positions of tokens in space, this information being loaded from the
\textit{Geometry}, or appearance information, loaded from the \textit{Appearance}. After
initialization, the Graphical Simulator displays the state of the simulation and handles all the
users' interaction as specified in the rest of this document.

For further details of what exactly each of the components allows the users to do please check the
following section or, in order to get more details, regarding implementation of \epns, please read the
\textit{System Specification}.

\subsection{General concepts}
\writer{Cosmin}
\label{oa:generalconcepts}
This subsection will introduce the general concepts used in the \epns system. More details are
provided in the \textit{System Specification}, however the most important concepts are presented below.

First of all, the classical concepts of \textbf{Petri nets} have been extended to accommodate the
required information for the graphical visualization of the simulation:
\index{Petri net} 
\begin{description}
\index{Petri net!Input Place}
\item[Input Places] - in order to provide the users with more power and customizability, some of the
Places in the Petri net can be configured to allow users, during the graphical simulation, to drop
(create) tokens. These are called \textit{Input Places} and act as normal Places in all other
respects, except for that they permit the possibility of a token being created there. For example,
in a train track simmulation, it allows the creation of simulation features such as a Traffic lights
or switches with which the users can interact during the graphical simulation.
\index{animation}
\item[Animations] - can be associated by the user to a particular Place and are run when a Token is
added on that Place, either by result of executing a transition or by being dropped there (after a
user interaction). Even though the token is removed from the source Places of the fired Transition,
they are not available for firing a new Transition until the animations associated with a place are
finished. More details will be specified later, but the supported animation types include: moving an
object on a path, showing or hiding objects, wait a fixed amount of time.
\item[Place Appearance] - each Place can have an associated appearance, describing how it must look
like in the Graphical Simulator.
\item[Token Appearance] - each Token can have an associated appearance, describing how it must look
like in the Graphical Simulator. Thus, the appearance of a Token will not change based on a place.
This will allow multiple tokens, with different representations, to be on the same place/track.
\item[Arc Identities] - each arc can have attached an identity used to control the flow of tokens
(or, more precisely, of token representations) in the simulation. For e.g., if we have a Transition
with one input Arc and two output Arcs and we take the case of simulating a train running on a
track, using the same identity on the input Arc and on one of the ouput Arcs will tell the Graphical
Simulator to move the Token representation (a train), which came on the input Arc, on the
corresponding output Arc. This allows a token representation to move continously in the direction
the user wants, without being destroyed or unnecessarily recreated.
\end{description}

Regarding the \textbf{Geometry}, as defined, it allow the users to specify the positions of objects
and the paths on which they move in the simulation space. The most important related concepts that
need to be presented at this point are:
\index{geometry}
\begin{description}
\index{Petri net!Track}
\item[Track] - defines how a curve (or line), on which an animation can take place, looks like. It
can also be connected with information about what the surface of this track looks like and usually
are used as graphical representations of Places.
\index{Petri net!Simple Position}
\item[Simple Position] - defines just a position in the simulation space and can be connected to the
an appearance it has. Can be used for completing the specification of some animations, for
representing an Input Place or just for displaying simple objects during the simulation.
\end{description}

Referring to the \textbf{Appearance}, it allow the users to easily define how objects look like
during the simulation. In \epns, there are mainly two big types of appearances that can be
configured:
\index{appearance}
\begin{description}
\item[Shape] - defines how a 3D Object displayed in the simulation should look like. For example, it
can be a reference to a file storing a 3D Model, which can then be loaded in the application or it
can simply be a 3D Object, such as a Cube or Sphere.
\item[Surface] - defines how a surface displayed in the simulation should look like. It can be
applied, for example, to a train track, and it could be either just a Color or a reference to a file
containing a texture that can be applied on the surface.
\end{description}


